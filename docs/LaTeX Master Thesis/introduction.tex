
Per Previous Teams:
\newline
The University of California, Santa Cruz’s Arboretum has a very neglected
micro-grid with practically non-existent management. This has enabled reconfiguration
by those who don’t understand the intricacies of their actions, and
hence encourages increasing complexity of the system. The grid itself is in a
terrible state of disarray, featuring unfortunate developments such as random,
hanging live wires, batteries more than 5 years expired, inadequate power for
more crucial items, and plenty of broken equipment with obfuscated wiring,
that still sit in the same housing as the working equipment. We are the team
tasked with amending this poor facility’s design once and for all, with great
advancements such as a completely new design, fool-proof safety, and guidelines
for future expansion. We hope this document elaborates on the dire need to
make changes to the grid as well as documents what changes we recommend
and plan to implement.
\newline

The monitoring project is part of the greenhouse research facility at the University of California, Santa
Cruz which aims to provide an interactive electrical system for producing successful harvests. It is an
ongoing project that is in its second iteration, 2017-18’, with the goal of remotely monitoring and
regulating the greenhouse environmental conditions. The client’s needs are to create a plug-n-play
metering hub and web interface for students to conduct horticulture experiments using solar energy.
This year’s productivity involved gathering old documentation, revamping the server, developing a usermanual,
and creating a website. The ability to monitor the electrical grid through serial communication
was researched, forwarded, and documented. To complete the prototype of the metering system there
must be power-data analytics on the website along with charts and graphs of conditions. This document
serves to assist in understanding the current state of the system and for future work.
